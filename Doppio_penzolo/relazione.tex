\documentclass{exam}
\date{30 Febbraio 2017}
\usepackage[italian]{babel}
\usepackage[T1]{fontenc}
\usepackage{graphicx}
\title{Pendolo quadrifilare}
\author{Francesco Sacco, Francesco Tarantelli, Giovanni Sucameli}
\usepackage{amsmath}
\usepackage{mathtools}
\usepackage{booktabs}
\author{Francesco Tarantelli, Francesco Sacco, Giovanni Sucamelo}
\title{Oscillazioni accoppiate}

\begin{document}
	\maketitle
	\section{Scopo dell'esperienza}
		lo scopo di questa esperienza \'e lo studi del moto di due pendoli accoppiati e, in particolare, del fenomeno dei battimenti


	\section{Cenni Teorici}
		\subsection{Pendolo singolo}
			In questa prima parte si cerca di verificare semplicemente che la pulsazione angolare $\omega_o$ del pendolo fisico senza attrito sia uguale a 
			\begin{equation}
				\omega_o=\sqrt{\frac{mgl}{I}}
			\end{equation}
			In seguito con lo smorzatore si \'e stimato il decadimento $\tau $ dell'ampiezza si oscillazione
			\begin{equation}
				\theta_o(t)=\theta_o(0)e^{-\frac{t}{\tau}}
			\end{equation}
	\subsection{Oscillazioni in fase e in controfase}
		Nelle oscillazioni in fase e in controfase si \'e in sostanza verificato l'equazione del moto dei pendoli nei due modi normali ottenuti dal sistema per un pendolo semplice:

		\begin{equation}
			\begin{cases} 
				m \ddot{\theta_{1}}=-\frac{mg}{l}\theta_1 + k(\theta_2-\theta_1) -\frac{m}{\tau}\dot{\theta_1} \\
				m \ddot{\theta_{2}}=-\frac{mg}{l}\theta_2 -  k(\theta_2-\theta_1) - \frac{m}{\tau} \dot{\theta_{2}}
			\end{cases}
		\end{equation}

		che equivale a:

		\begin{equation}
			\begin{bmatrix}
				m & 0 \\
				0 & m
			\end{bmatrix}
			\mathbf{q''}=-
			\begin{bmatrix}
				\frac{mg}{l} + k & -k \\
				-k & \frac{mg}{l} + k \\
			\end{bmatrix}
			\mathbf{q} -
			\begin{bmatrix}
				\frac{m}{\tau} & 0 \\
				0 & \frac{m}{\tau} 
			\end{bmatrix}
			\mathbf{q'} 
		\end{equation}

		dove q=$
		\begin{bmatrix}
			\theta_1 \\
			\theta_2
		\end{bmatrix}$
		. La soluzione generale di questa equzione pu\'o essere scritta nella forma:
		\begin{equation}
			\label{eq1}
			\theta(t)= A_0 e^{-\frac{t}{\tau}}[\cos(t\omega_f + \phi_1) +\sin(t\omega_c + \phi_2) ]
		\end{equation}
		in particolare trascurando l'attrito, $\omega_f$ e $\omega_c$ sono uguali alle pulsioni angolari dei modi normali\\($\omega_{f}^2=\frac{g}{l} \omega_{c}^2=\frac{g}{l}+2\frac{k}{m} $). L'equazione \ref{eq1} \'e molto importante perch\'e viene utilizzata per descrivere i battimenti.

	\section{Apparato sperimentale}
		\begin{itemize}
			\item Due pendoli
			\item Molla
			\item Due smorzatori
			\item Sistema di acquisizione
		\end{itemize}

	\section{Analisi dati}
		%Smorzato -------------------------------------------------------------------------
		\subsection{Pendolo smorzato}
			Per prima misurazione abbiamo analizzato il moto di un pendolo con galleggiante	e per trovare la costante di smorzamento $\tau_{0}$ \\
			\begin{minipage}{0.5\textwidth}
				\includegraphics[width=\textwidth]{fit_smorzato}
				\end{minipage}
			\begin{minipage}{0.5\textwidth}
				\begin{tabular}{ll}
					\toprule
					Dati & Parametri ottimali \\
					\midrule
					$\tau_{0}[\textrm{s}]$ & $16,24 \pm 0,02$ \\
					$A_{0}[\textrm{cm}]$ & $4,51 \pm 6,01(10^{-6})$\\
					$\omega_{0}[\textrm{s}^{-1}]$ & $4,42 \pm 2,7(10^{-7})$\\			
					$\phi_{0}$ & $3,94 \pm 3,16$\\
					\bottomrule

				\end{tabular}
			\end{minipage}
			Si osservi che i punti sperimentali non seguono perfettamente una curva esponenziale, poich\'e il modello teorico non tiene in considerazione distubi esterni come l'attrito del perno e rumore esterno, e a causa di ci\'o il chi quadro risulta enorme, tuttavia la precisione sull'ampiezza e sul periodo \'e comunque parecchio elevata

		%In fase ---------------------------------------------------------------------------
			\subsection{Pendoli in fase}
			In seguito abbiamo raccolto i dati degli oscillatori in fase, come si pu\'o notare $\omega_{1}$ che $\tau_{1}$ sono praticamente uguali a $\omega_{0}$ e $\tau_{0}$ , questo perch\'e la molla resta alla sua posizione di riposo e quindi \'e come se non ci fosse\\
			\begin{minipage}{0.5\textwidth}
				\includegraphics[width=\textwidth]{fase}
			\end{minipage}
			\begin{minipage}{0.5\textwidth}
				\begin{tabular}{ll}
					\toprule
					Dati & Parametri ottimali \\
					\midrule
					$\tau_{f}[\textrm{s}]$ & $15,72 \pm 0,02$ \\
					$A_{f}[\textrm{cm}]$ & $17,29 \pm 6,75(10^{-7})$\\
					$\omega_{f}[{\textrm{s}^{-1}}]$ & $4,17 \pm 2,41(10^{-5})$\\			
					$\phi_{f}$ & $4,45 \pm 2,63(10^{-7})$\\
					\bottomrule
				\end{tabular}
			\end{minipage}
			In questo grafico abbiamo traslato il centro dell'oscillazione a 0, perch\'e la molla spostava la posizione d'equilibrio verso l'altro pendolo, inoltre abbiamo messo solo il grafico di uno dei due pendoli, visto che inserire l'altro risultava ridondante

		%In controfase ---------------------------------------------------------------------
		\subsection{Pendoli in controfase}
			Prima di effettuare la misura dei battimenti abbiamo fatto quella dei pendoli in controfase cosicch\'e ottiniamo i valori di $\omega_{c}$ per verificare che ci\'o che \'e scritto nei cenni teorici\\
			\begin{minipage}{0.5\textwidth}
				\includegraphics[width=\textwidth]{controfase}
				\end{minipage}
			\begin{minipage}{0.5\textwidth}
				\begin{tabular}{ll}
					\toprule
					Dati & Parametri ottimali \\
					\midrule
					$\tau_{c}[\textrm{s}]$ & $17,27 \pm 0,03$ \\
					$A_{c}[\textrm{cm}]$ & $1,53 \pm 4,89(10^{-7})$\\
					$\omega_{c}[{\textrm{s}^{-1}}]$ & $6,51 \pm 2,11(10^{-5})$\\			
					$\phi_{c}$ & $4,61 \pm 2,87(10^{-7})$\\
					\bottomrule
				\end{tabular}
			\end{minipage}
			La prima cosa che salta all'occhio \'e che $\omega$ \'e aumentato come si ci aspettava, mentr $\tau$ non cambia di molto \'e
		%Battimenti ------------------------------------------------------------------------
		\subsection {Battimenti}
			Dulcis in fundu, abbiamo fatto la raccolta dati dei battimenti e fatto il fit. Questo fit \'e risultato parecchio impegnativo perch\'e sembrava non voler trovare il minimo $\chi^2$, ma alla fine cel'abbiamo fatta.\\
			\begin{minipage}{0.5\textwidth}
				\includegraphics[width=\textwidth]{Battimenti}
				\end{minipage}
			\begin{minipage}{0.5\textwidth}
				\begin{tabular}{ll}
					\toprule
					Dati & Parametri ottimali \\
					\midrule
					$\tau[\textrm{s}]$ & $64,30 \pm 0,09$ \\
					$A[\textrm{cm}]$ & $7,05 \pm 5,67(10^{-7})$\\
					$\omega_{a}[\textrm{s}^{-1}]$ & $4,51 \pm 4,70(10^{-9})$\\
					$\omega_{b}[\textrm{s}^{-1}]$ & $6,48 \pm 6,64(10^{-9})$\\			
					$\phi_{a}$ & $2,05 \pm 4,37(10^{-6})$\\
					$\phi_{b}$ & $3,19 \pm 9,50(10^{-6})$\\
					\bottomrule
				\end{tabular}
			\end{minipage}
			dalla lettura dei dati si ci accorge che $\omega_{a}$ \'e molto simile a $\omega_{f}$ e $\omega_{b}$ a $\omega_{c}$, ci\'o \'e previsto dalla teoria.
	\section{Conclusione}
		La raccolta dati ci conferma che il modello teorico \'e corretto anche se il $\chi^2$ risulta straordinariamente alto (nell'ordine dei milioni) e il p-value viene 0 spaccato


\end{document}