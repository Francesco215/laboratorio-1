\documentclass{exam}
\date{3 Aprile 2017}
\usepackage[italian]{babel}
\usepackage[T1]{fontenc}
\usepackage{graphicx}
\title{Pendolo quadrifilare}
\author{Francesco Sacco, Francesco Tarantelli, Giovanni Sucameli}
\usepackage{amsmath}
\usepackage{mathtools}
\usepackage{booktabs}
%\DeclarePairedDelimiter{\abs}{\lvert}{\rvert}
%\DeclarePairedDelimiter{\norma}{\lVert}{\rVert}
\author{Francesco Tarantelli, Francesco Sacco, Giovanni Sucamelo}
\title{Oscillazioni accoppiate}

\begin{document}

	\maketitle



	\section{Cenni Teorici}
		\subsection{Pendolo singolo}
			In questa prima parte si cerca di verificare semplicemente che la pulsazione angolare $\omega_o$ del pendolo fisico senza attrito sia uguale a 
			\begin{equation}
				\omega_o=\sqrt{\frac{mgl}{I}}
			\end{equation}
			In seguito con lo smorzatore si è stimato il decadimento $\tau $ dell'ampiezza si oscillazione
			\begin{equation}
				\theta_o(t)=\theta_o(0)e^{-\frac{t}{\tau}}
			\end{equation}
	\subsection{Oscillazioni in fase e in controfase}
		Nelle oscillazioni in fase e in controfase si \'e in sostanza verificato l'equazione del moto dei pendoli nei due modi normali ottenuti dal sistema per un pendolo semplice:

		\[
			\begin{cases} 
				m x''_1=-\frac{mg}{l}x_1 + k(x_2-x_1) -\frac{m}{\tau} x'_1 \\
				m x''_2=-\frac{mg}{l}x_2 -  k(x_2-x_1) - \frac{m}{\tau} x'_2
			\end{cases}
		\]

		che equivale a:

		\[
			\begin{bmatrix}
				m & 0 \\
				0 & m
			\end{bmatrix}
			\mathbf{q''}=-
			\begin{bmatrix}
				\frac{mg}{l} + k & -k \\
				-k & \frac{mg}{l} + k \\
			\end{bmatrix}
			\mathbf{q} -
			\begin{bmatrix}
				\frac{m}{\tau} & 0 \\
				0 & \frac{m}{\tau} 
			\end{bmatrix}
			\mathbf{q'} 
		\]

		dove q=$
		\begin{bmatrix}
			x_1 \\
			x_2
		\end{bmatrix}$
		. La soluzione generale di questa equzione pu\'o essere scritta nella forma:
		\begin{equation}
			\label{eq1}
			x(t)= A_0 e^{-\frac{t}{\tau}}[\cos(\omega_1 t + \phi_1) +\sin(\omega_2 t + \phi_2) ]
		\end{equation}
		in particolare trascurando l'attrito, $\omega_1$ e $\omega_2$ sono uguali alle pulsioni angolari dei modi normali\\($\omega_{fase}^2=\frac{g}{l} \omega_{contro}^2=\frac{g}{l}+2\frac{k}{m} $). L'equazione \ref{eq1} \'e molto importante perch\'e viene utilizzata per descrivere i battimenti.



\end{document}